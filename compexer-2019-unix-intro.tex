\documentclass[dvipdfmx]{beamer}
\usepackage{setspace}
\usepackage{url}

\AtBeginDvi{\special{pdf:tounicode EUC-UCS2}}

\usetheme{metropolis}
\setbeamertemplate{footline}[frame number]\renewcommand{\kanjifamilydefault}{\gtdefault}
\renewcommand{\kanjifamilydefault}{\gtdefault}

\title{Unix と Linux の紹介}
\author{照井 章}
\institute{筑波大学 数理物質系}
\date{2019年7月23日}

\begin{document}
    
\begin{frame}
    \frametitle{}
    \titlepage

    \begin{center}
        \url{https://github.com/tsukuba-compexer-2019/compexer-2019-unix-intro}
    \end{center}
\end{frame}

\begin{frame}
    \frametitle{この話の内容}
    \large
    \setstretch{1.5}
    \begin{itemize}
        \item Unix, Linux の生い立ちや特徴など
    \end{itemize}
\end{frame}

\begin{frame}
    \frametitle{Unix とは? }

    \begin{block}{オペレーティングシステム (Operating System, OS)}
        \begin{itemize}
            \item 計算機上で動作するソフトウェア
            \item 入出力装置や記憶装置とのデータのやり取りを管理する
            \item 計算機上で動作する他のソフトウェアの動作を管理する
            \item 例: Windows 10 (Microsoft), macOS (Apple), iOS (Apple), Android (Google) など
        \end{itemize}
    \end{block}
\end{frame}

\begin{frame}
    \frametitle{Unix の生い立ち---ゲームから生まれた!? \cite{spacetravel}}
    \begin{itemize}
        \item<1-> 1969年、AT\&T ベル研究所にて誕生。主な開発者は Ken Thompson, Dennis Ritchie \cite{dmr} ら
        \begin{itemize}
            \item<2-> 当時、MITのプロジェクトにベル研が参加し、MulticsというOSを開発していた
            \item<3-> しかし、プロジェクトの不振によりベル研がMulticsから撤退
            \item<4-> Multicsで ``Space Travel'' というゲームをやっていた彼らは、そのゲームができなくなることを惜しんだ(らしい)
            \item<5-> その時、自分達の職場に小さなコンピュータ (PDP-7) が放置されているのを発見
            \item<6-> 「そいつにゲームを移植しよう!」\dots Unixの始まり
        \end{itemize}
    \end{itemize}
\end{frame}

\begin{frame}
    \frametitle{Unixの普及}
    \begin{itemize}
        \item 1970年代から'80年代にかけて、世界の大学や研究機関を中心に普及
        \item 日本には、石田晴久先生(故人)が1975年頃にベル研から日本に持ち帰る。最初にインストールされたのは筑波大学のマシンだったらしい \cite{sunahara-murai}
        \item さらに産業界にも普及
        \item 1990年代から、Unix系OS (BSD, Linux等) が多数誕生し、普及する
        \begin{itemize}
            \item AppleのmacOSはDarwinというカーネルに基づくUnix系OS
        \end{itemize}
        \item 1080年代終盤からはOSの国際標準の策定にもUnixが貢献
    \end{itemize}
\end{frame}

\begin{frame}
    \frametitle{Unixはなぜ普及したか?}

    \begin{block}{Unix以前にこれだけ多種多様なハードウェアに移植されたOSはほとんどなかった}
        \begin{itemize}
            \item OSの主要部がC言語で書かれていて移植性に優れていた
            \item 単純で柔軟なファイル構造によに、複雑な処理が可能
            \item ほぼ無償(実費程度の費用)で配布された
            \item シェルによる対話的な操作性に優れていた
            \item 開発ツールの使いやすさ
            \item ネットワーク関連機能の充実(これが後に、Unixがインターネットの各種サーバとして使われることにつながる)
        \end{itemize}
    \end{block}
\end{frame}

\begin{frame}
    \frametitle{Linux とは? \cite{linuxfoundation}}

    \begin{block}{Unix系OSの一つ}
        \begin{itemize}
            \item ``Linux'' は主に2つの意味を持つ
            \begin{itemize}
                \item Linux カーネル (Kernel): OSの中核部分
                \item Linux ディストリビューション (Distribution): カーネル + ライブラリ + 管理ソフトウェア + アプリケーションソフトウェア 
            \end{itemize}
            \item Linux カーネルの生みの親: Linus Tovalds (Gitの作者)
        \end{itemize}
    \end{block}
\end{frame}

\begin{frame}
    \frametitle{Linuxはなぜ普及したか?}

    \begin{block}{ハードウェアの性能向上とオープンなソフトウェアライセンス/開発形態}
        \begin{itemize}
            \item 1990年代に入り、パソコンの性能が大幅に進歩かつ低価格化し、Unix系OSが個人にも十分な実用性で動作する程度になった
            \begin{itemize}
                \item CPU: Intel, AMD / クロック周波数 数百MHz
                \item メモリ: 数十MB
                \item ハードディスク: 数十〜数百MB
            \end{itemize}
            \item 1990年代前半において、法的その他の問題がなく、個人が自由に使え、機能や性能が本格的なUnix系OSはLinuxくらいだった
            \item Linux はフリーソフトウェアのライセンスを導入するとともに、個人や産業界の開発者を広く受け入れ、開発者が増えた
        \end{itemize}
    \end{block}
\end{frame}

\begin{frame}%[allowframebreaks]
    \frametitle{筆者の学生時代のUnix/Linux等とのかかわり(学類)}
    \begin{itemize}
        \item<1-> 1991年初め: 大学受験期にUnix (MINIX) の本を読んで妄想
        \item<2-> 1991年春: 筑波大学入学。学内で使えるUnixを探してさまよう。情報学類(現・情報科学類)の計算機システム (coins) の利用許可を得る
        \item<3-> 1992年春: 大学の教育用システム(現・全学計算機システム)にUnixが導入される。端末にはWindows 3.1とMacintoshが入る。
        \item<4-> 1992年夏: 数学外書輪講のレポートに初めて \LaTeX を使う。同級生とMathematicaの自主ゼミを開く。
        \item<5-> 1993年夏: 教職科目(教育情報処理)の受講をきっかけに、Unix 管理に関わる。
    \end{itemize}
\end{frame}

\begin{frame}%[allowframebreaks]
    \frametitle{筆者の学生時代のUnix/Linux等とのかかわり(大学院)}
    \begin{itemize}
        \item<1-> 1995年春: 大学院に進学。春先から数学系(現・数学域)のUnixサーバの管理に関わる。生物資源学類 (bres) の計算機システム立ち上げに関わる。
        \item<2-> 1995年夏頃: 研究室にLinuxシステムを導入。
        \item<3-> 1999年秋: 数学系助手になる。
        \item<4-> 2000年: 数学系メールサーバにLinuxシステムを導入。教員と大学院生の共同による管理体制を組織する。
    \end{itemize}
\end{frame}

\begin{frame}[allowframebreaks]
    \frametitle{参考文献}
    \begin{thebibliography}{99}

        \bibitem{linuxfoundation} The Linux Foundation. \url{https://www.linuxfoundation.org/} (参照 2019-07-22).

        \bibitem{dmr} Dennis M. Ritchie. \url{http://cm.bell-labs.co/who/dmr/index.html} (参照 2019-07-21).

        \bibitem{spacetravel} Dennis M. Ritchie. Space Travel: Exploring the solar system and the PDP-7. \url{http://cm.bell-labs.co/who/dmr/spacetravel.html} (参照 2019-07-21).
        
        \bibitem{sunahara-murai} 砂原秀樹, 村井純. 石田先生から受け継いだもの. 情報処理, Vol.50, No.7, July 2009. \url{http://id.nii.ac.jp/1001/00060808/}

        
    \end{thebibliography}    
\end{frame}

\end{document}